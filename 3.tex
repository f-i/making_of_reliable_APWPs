\begin{summary}
Using our APWP similarity measuring tool to find which method\(s\) is$(are)$
good or bad.
\end{summary}

\begin{keywords}
  Moving Average \textendash{} Weighting \textendash{} APWP \textendash{}
  Paleomagnetism.
\end{keywords}

\section{Introduction}

APWPs are generated by combining paleomagnetic poles for a particular rigid
block over the desired age range to produce a smoothed path.

\subsection{Not All Data Are Created Equal}

However, uncertainties in the age and location of paleomagnetic poles can vary
greatly for different poles.

\subsubsection{Age Error}

Although remanent magnetizations are generally assumed to be primary, many
events can cause remagnetisation (in which case the derived pole is `younger'
than the rock). If an event that has occurred since the rock's formation that
should affect the magnetisation (e.g., folding, thermal overprinting due to
intrusion) can be shown to have affected it, then it constrains the
magnetisation to have been acquired before that event. Recognising or ruling
out remagnetisations depends on these field tests, which are not always
performed or possible. Even a passed field test may not be useful if field test
shows magnetisation acquired prior to a folding event tens of millions of years
after initial rock formation.

The most obvious characteristic we can observe from paleomagnetic data is that
some poles have very large age ranges, e.g., more than 100 Myr. The
magnetization age should be some time between the information of the rock and
folding events. There are also others where we have similar position but the
age constraint is much narrower, e.g. 10 Myr window or less. Obviously the
latter kind of data is more valuable than the one with large age range.

\subsubsection{Position Error}

The errors of pole latitudes and longitudes are 95\% confidence ellipses,
which also vary greatly in magnitude. All paleomagnetic poles have some
associated uncertainties due to measurement error and the nature of the
geomagnetic field. More uncertainties can be added by too few samples,
sampling spanning too short a time range to approximate a GAD field, failure
to remove overprints during demagnetisation, etc.

\subsubsection{Data Consistency}

Paleomagnetic poles of a rigid plate or block should be continuous time series.
For a rigid plate, two poles with similar ages shouldn't be dramatically
different in location. Sometimes, this is the case. Sometimes we have further
separated poles with close ages.

There are a number of possible causes for these outliers, including:

\verb"Lithology"

For poor consistency of data, it is potentially because of different
inclinations or declinations. The first thing we should consider about is
their lithology. We want to check if the sample rock are igneous or
sedimentary, because sediment compaction can result in anomalously shallow
inclinations~\cite{T18}. In addition, we also can check if the rock are
redbeds or non-redbeds. Although whether redbeds record a detrital signal or a
later Chemical Remanent Magnetization (CRM) is still somewhat controversial,
both sedimentary rocks and redbeds could lead to inconsistency in direction
compared to igneous rocks.

\verb"Local Rotations"

Local deformation between two paleomagnetic localities invalidates the rigid
plate assumption and could lead to inconsistent VGP directions. So if
discordance is due to local deformation, and we would ideally want to exclude
such poles from our APWP calculation.

\verb"Other Factors"

In most cases, mean pole age (centre of age error) has just been binned. If
any of the poles have large age errors, they could be different ages from each
other and sample entirely different parts of the APWP\@. Conversely, if any of
the poles have too few samples, or were not sampled over enough time to average
to a GAD field, a discordant pole may be due to unreduced secular variation.

\subsubsection{Data Density}

As we go back in time, we have lower quality and lower density (or quantity) of
data, for example, the Precambrian or Early Paleozoic paleomagnetic data are
relatively fewer than Middle-Late Phanerozoic ones, and most of them are not
high-quality, e.g., larger errors in both age and location. The combination of
lower data quality with lower data density means that a single `bad' pole (with
large errors in age and/or location) can much more easily distort the
reconstructed APWP, because there are few or no `good' poles to counteract its
influence.

Data density also varies between different plates. E.g., we have a relatively
high density of paleomagnetic data for North American Craton (NAC), but few
poles exist for Greenland and Arabia. Based on mean age (mean of lower and
upper magnetic ages), for 120\textendash0 Ma, the \textbf{Global Paleomagnetic
Database} (GPMDB) version 4.6b~\cite{P05} has more than 130 poles for NAC, but
only 17 for Greenland and 24 for Arabia.

\subsubsection{Publication Year}

The time when the data was published should also be considered, because
magnetism measuring methodology, technology and equipments have been improved
since the early 20th century. For example, stepwise demagnetisation, which is
the most reliable method of detecting and removing secondary overprints, has
only been in common use since the mid 1980s.

In summary, not all paleomagnetic poles are created equal, which leads to an
important question: how to best combine poles of varying quality into a
coherent and accurate APWP\@?

\subsection{Existing Solutions and General Issues}

Paleomagnetists have proposed a variety of methods to filter so-called ``bad''
data, or give lower weights to those ``bad'' data before generating an APWP,
e.g., two widely used methods: the V90 reliability criteria~\cite{v90} and the
BC02 selection criteria provided by Besse \& Courtillot \shortcite{B02}.
Briefly, the V90 criteria for paleomagnetic results includes seven criteria: (1)
Well determined age; (2) At least 25 samples with Fisher~\cite{F53} precision
$\kappa$ greater than 10 and $\alpha95$ less than 16\degree; (3) Detailed
demagnetisation results reported; (4) Passed field tests; (5) Tectonic coherence
with continent and good structural control; (6) Identified antipodal reversals;
(7) Lack of similarity with younger poles~\cite{T92}. The total criteria
satisfied (0\textendash7) is then used as a measure of a paleomagnetic result's
overall reliability, which is known as Q (quality) factor~\cite{T92}. Q factor
is indeed a very straightforward way to get a quantitalized reliability score.
Also it then can be conveniently used in the later calculations of
APWPs~\cite{T92}. But at the same time this is a fairly basic filter that lumps
together criteria that may not be equally important. Compared with V90, the BC02
criteria suggests stricter filtering, e.g., using only poles with at least 6
sampling sites and 36 samples, each site having $\alpha95$ less than 10\degree\
in the Cenozoic and 15\degree\ in the Mesozoic. B02 is also straightforward and
convenient to use, but some useful data may be filtered out and wasted
especially for a period where there are only limited number of data. In
addition, there has been limited study of how effective these marking/filtering
methods are at reconstructing a `true' APWP, and for most studies after a basic
filtering of `low quality' poles, the remaining poles are, in fact, treated
equally.

Above all, there haven't been any real attempts to study how APWP fits may be
improved by filtering/weighting data. This paper is presented to address these
issues.


\section{Methods}

For most of Earth history, concretely for times before c. 170 Ma, the age of the
oldest magnetic anomaly identification, paleomagnetism is the only accepted
quantitative method for reconstructing plate motions and past paleogeographies.
After about 170 Ma, multiple data sources can help constrain plate motions in
more accurate ways. One of the most developed and studied plate kinematics
models is the Fixed Hotspot Model (FHM)~\cite{M93,M99}, which assumes the
Atlantic and Indian hotspots are relatively fixed. Such a model like the Fixed
Hotspot can predict APWPs for main continents, e.g.\ the North America.

\subsection{Reference Path: The Hotspot Model Predicted}

The oldest pole that can be predicted from the FHM is about 120 Ma. The North
American 120\textendash0 Ma APWP predicted from this rotation model and
latest published spreading ridge rotations (collected data will be shared as a
supplementary material) will be taken as a reference path
(Fig.~\ref{fig-fhsPred}), which will be compared with paleomagnetic APWPs for
the same plate or continent.

\begin{figure*}
\centering
\includegraphics[width=1.01\textwidth]{figures/fhs.pdf}
\caption[120\textendash0 Ma FHS model predicted APWP of North America]{FHM
predicted 120\textendash0 Ma APWP for $NAC$. Its age step is 5
Myr.}\label{fig-fhsPred}
\end{figure*}

\subsection{120\textendash0 Ma North American Paleomagnetic APWP}

The GPMDB 4.6b~\cite{P05}, data source used here, includes 9514 paleopoles for
ages of 3,500 Ma to the present published from 1925 to 2016. A polygon can be
drawn around a set of data, whose sampling sites we believe belong to a specific
plate or rigid block. Then the {\em Spatial Join\/} technique~\cite{J07} helps
join attributes from the polygon to the paleomagnetic data based on the spatial
relationship allowing data within this polygon to be extracted from the whole
raw large dataset without splitting a subset just for a specific plate. That
allows us to quickly select subsets of the database based on geographic
constraints just as easily as for age. Of course, the boundary of this polygon
must be reasonably along a tectonic boundary (see the details about data
filtering for North America in the Supplementary Material). The temporal
distribution of North American 120\textendash0 Ma poles is shown in
Fig.~\ref{fig-120NAhist}.

\begin{figure*}
\centering
\includegraphics[width=1.01\textwidth]{figures/120NAhist.pdf}
\caption[Distribution of 120\textendash0 Ma North American poles]{Temporal
distribution 120\textendash0 Ma $NAC$ paleomagnetic poles. For distribution a,
each bin (spanning 10 Myr here) only counts in the midpoints of pole error bars;
For distribution b, as long as the bar intersects with the bin, it is counted
in.}\label{fig-120NAhist}
\end{figure*}

\subsection{Picking Data for A Certain Time Window}

\subsubsection{Moving Average}

The moving average method, also called ``running mean'' or ``moving
window''~\cite{T08} method, calculates the average of values between a certain
data (age in our case) range; the average is then recalculated as the limits of
the bin are repeatedly incremented upwards. In addition to the traditional
moving windows averaging algorithm, a newly developed moving average method is
also used, referred to here as the ``Age Position Picking (APP)'' method. The
difference of this moving average method from the one built in
GMAP~\cite{T99,T08} is that the whole magnetic age range is taken into account
in each window, while GMAP only considers the mid-point of the low and high
magnetic age of each pole, an algorithm referred to as the ``Age Mean Picking
(AMP)'' method.

Normally each VGP in the paleomagnetic database is treated as a point with an
age that is the mid-point between the upper and lower age limits, i.e. AMP, but
this is problematic for paleomagnetic data with large age ranges (especially if
they turn out to be primary magnetization that should plot at old end of age
range). We are trying a method, APP, that includes a VGP in the moving average
bin if any part of its specified age range falls within that bin. If, for
example, we have a pole which is constrained to within 10 and 20 Ma of age, and
we have a 2 Myr moving window with a 1 Myr age step, then it shouldn't just be
in the 14\textendash16 Ma bin (for the mid-point age of 15 Ma)\textemdash{}it
should be in the 9\textendash11,10\textendash12,11\textendash13,12\textendash14
\ldots17\textendash19,18\textendash20, and 19\textendash21 Ma bins. So the
average poles are produced from each bin, and each original pole is represented
over its entire possible acquisition age. Fig.~\ref{fig-nac-maplat} shows an
example of moving average with a 10 Myr window and a 5 Myr step. So, for
example, for the window of 15 Ma to 5 Ma (the light blue bin in
Fig.~\ref{fig-nac-maplat}), the AMP method calculates the Fisher mean pole of
only 5 poles, while the APP method calculates the mean pole of 9 poles. From
comparison of mean poles of the picked poles for the light blue age window with
the two different algorithms (the 10 Ma mean poles in Fig.~\ref{fig-fhsPred}),
the mean pole from the APP method is closer to the 10 Ma pole in the FHM
predicted path.

\begin{figure*}
\centering
\includegraphics[width=1.01\textwidth]{figures/binning.pdf}
\caption[Moving average (MA) methods]{An example of 10 Myr moving window and 5
Myr step in the moving average method, based on poles of the $NAC$. Every age
window has a different color. Red points are the midpoints of low and high
magnetic ages. The vertical axis has no specific meaning here.
}\label{fig-nac-maplat}
\end{figure*}

\paragraph{Picking}

The 28 picking methods include AMP, APP and also those with filtering or
corrections implemented onto the two (Table.~\ref{tab-pick}).

\begin{table}
\centering
\caption{List of all Picking (i.e. Binning) algorithms developed in this paper.
         AMP, Age Mean Picking (See Section ``Moving Average''); APP, Age
         Position Picking.}\label{tab-pick}
\begin{tabular}{@{}ll@{}}
\toprule
No. & Picking Algorithm \\ \midrule
0 & AMP \\
1 & APP \\
2<18 & AMP (``$\alpha$95/Age range'' no more than ``15/20'') \\
3<2 & APP (``$\alpha$95/Age range'' no more than ``15/20'') \\
4<16 & AMP (mainly or only igneous) \\
5<3 & APP (mainly or only igneous) \\
6<17 & AMP (contain igneous and not necessarily mainly) \\
7<4 & APP (contain igneous and not necessarily mainly) \\
8<12 & AMP (unflatten sedimentary) \\
9<5 & APP (unflatten sedimentary) \\
10<19 & AMP (nonredbeds) \\
11<6 & APP (nonredbeds) \\
12<20 & AMP (unflatten redbeds) \\
13<7 & APP (unflatten redbeds) \\
14<21 & AMP (published after 1983) \\
15<8 & APP (published after 1983) \\
16<22 & AMP (published before 1983) \\
17<9 & APP (published before 1983) \\
18<23 & AMP (exclude commented local rot or secondary print) \\
19<10 & APP (exclude commented local rot or secondary print) \\
20<24 & AMP (exclude local rot or correct it if suggested) \\
21<11 & APP (exclude local rot or correct it if suggested) \\
22<25 & AMP (filtered using SS05 palaeomagnetic reliability criteria) \\
23<13 & APP (filtered using SS05 palaeomagnetic reliability criteria) \\
24<26 & AMP (exclude superseded data already included in other results) \\
25<14 & APP (exclude superseded data already included in other results) \\
26<27 & AMP (comb of 25 and 26) \\
27<15 & APP (comb of 13 and 14) \\ \bottomrule
\end{tabular}
\raggedright{Notes: SS05,~\cite{S05}}
\end{table}

\subparagraph{Filtering through $\alpha$95 and Age Range}

For this specific filter, the poles are picked out through setting $\alpha$95 of
$\leq15\degree$ and age uncertainty of $\leq20$ Myr.

\subparagraph{Filtering Out Non-igneous Derived Poles}

With this filter, the poles are mainly or only from igneous rocks with extrusive
or intrusive type.

\subparagraph{Filtering Out Igneous Unrelated Poles}

With this filter, the poles are from rocks that contain extrusive or intrusive
igneous type. In other words, the rock type could be mainly sedimentary or
metamorphic.

\subparagraph{Inclination Shallowing and Unflattening}

To test if unflattening possible inclination shallowing in sedimentary rocks can
improve the APWP fitting outcomes, the flattening function~\cite{K55} $\tan I_o
= f \tan I_f$ is used to unflatten assumed existing inclination shallowing in
sedimentary-based or redbeds-based paleomagnetic data (No $5$ and $7$ in
Table.~\ref{tab-pick}), where $I_o$ is the observed inclination, $I_f$ is the
unflattened inclination, and $f$ is the flattening factor (or shallowing
coefficient) ranging from unity (no flattening) to 0 (completely flattened).
Here $f=0.6$ is used in our calculations, according to the previous
experience~\cite{T12}.

\subparagraph{Filtering Out Poles Related to Red-beds}

Bias toward shallow inclinations is also observed in paleomagnetic data derived
from red-beds~\cite{T04}. For this filter, the poles derived from red-beds are
simply removed.

\subparagraph{Filtering Out Poles Published Earlier or Later}

It is also worthy to see if recently published data is able to produce a more
reliable APWP than relatively older data. Here 1983 is chosen as the division,
because the mean of the data publication years is about 1983.

\subparagraph{Filtering Out Poles Influenced by Local Rotations or Secondary
Print}

Some publications of paleomagnetic data suggest the data has probably been
affected by local rotations or secondary overprints. So for this filter, this
type of data are removed.

\subparagraph{Filtering Out Poles Influenced by Local Rotations or Correcting
Them if Suggested}

Some publications suggest the data has been through local rotation and propose a
solution of correction. With this filter, if there is a correction suggestion,
the data is corrected; if no, the data is simply removed.

\subparagraph{Filtering Using SS05 Liability Criteria}

SS05~\cite{S05} provided their criteria of picking paleomagnetic data for
producing their APWPs. This filter is reproducing their criteria by setting
$\alpha$95 of $\leq15\degree$, age uncertainty of $\leq40$ Myr, sampling sites'
quantity of $\geq4$, samples' quantity of at least 4 times of the sites, and
laboratory analytical procedure code of at least 2.

\subparagraph{Filtering Out Superseded Data}

In this filter, those superseded data already included in other newer results
are excluded.

\paragraph{Weighting}

Because all data is not created equal, we want to calculate a weighted mean pole
for a time interval with `better' (more likely to be reliable) poles counting
more than `worse'. For example, a pole with small $\alpha$95 and very well
constrained age is more likely to reflect APWP position at the selected age
point than a pole with large $\alpha$95 and very broad age range. There are many
potential ways to weight this data set which can obviously greatly influence the
final result, and we want to test this.

Six weighting algorithms (Table.~\ref{tab-weit}) have currently been developed
or reproduced according to published work to give different weights to data with
different qualities.

\begin{table}
\centering
\caption{List of all weighting algorithms developed in this
         paper.}\label{tab-weit}
\begin{tabular}{@{}ll@{}}
\toprule
No. & Weighting Algorithm \\ \midrule
0 & None \\
1 & Numbers of sites (B), Observations (N) \\
2 & Age uncertainty \\
3 & $\alpha$95 \\
4 & Age error Position to bin \\
5 & comb of 3 and 4 \\ \bottomrule
\end{tabular}
\end{table}

In order to average errors in orientation of the samples and scatter caused by
secular variation, a ``sufficient'' number of individually oriented samples
(observations) from ``enough'' sites must be satisfied~\cite{T18,v90,B02}. So
for the ``Numbers of sites (B), Observations (N)'' weighting (No $1$ in
Table.~\ref{tab-weit}), larger B and N mean stronger weighting. Through knowing
the pattern of all B and N in the database, the proposed solutions are as
follows. If both B and N are more than 1, weight=$(1- \frac{1}{B})*(1-
\frac{1}{N})$. There are data in GPMDB with only the number of sampling sites
(at least greater than 1) given, but no number of samples or only one sample
given, so for this case, if B$>$1 and N $\leq$ 1, weight=$(1- \frac{1}{B})*0.5$.
If only the number of samples (at least greater than 1) is given, and the
number of sampling sites is missing or only one, i.e. B$\leq$ 1 and N$>$1,
weight=$(1- \frac{1}{N})*0.5$. If B $\leq$ 1 and N $\leq$ 1 (there are only 23
datasets for the whole GPMDB 4.6b, including 18 with both B and N informations
missing), weight=0.2.

As for the ``Age uncertainty'' weighting (No $2$ in Table.~\ref{tab-weit}), a
well-constrained age should be known to within a half of a geological period
(e.g., Quaternary, Neogene, Triassic) for Phanerozoic data~\cite{v90,T18}.
Generally, this work follows this principle. However, for the periods of
Paleogene, Cretaceous, and Jurassic, their halves are all beyond a time span of
at least 20 Myr, which is relatively large for these relatively young geologic
periods. So for these three periods, a tighter age constraint is set using age
uncertainties of $\leq15$ Myr. So, for example, for NAC's Neogene
(23.03\textendash2.58 Ma according to GSA Geologic Time Scale) data, if age
uncertainty (the high magnetic age $-$ the low magnetic age) $\leq$ 10.225 (from
0.5* (23.03$-$2.58)) Myr, its weight = 1; if age uncertainty $>$ 10.225 Myr, its
weight = 10.225 / (high magnetic age $-$ low magnetic age). For the periods
spanning Jurassic to Paleogene, if age uncertainty $\leq15$ Myr, it get its
weight of 1; if age uncertainty $>15$ Myr, a weight of 15 / (high magnetic age
$-$ low magnetic age) is assigned instead.

For the ``$\alpha$95'' weighting (No $3$ in Table.~\ref{tab-weit}), smaller
radius of circle of 95\% confidence about mean remanence direction means less
error, so should get larger weight. Here, weight is from a Gaussian distribution
centered on 0 with standard deviation of 10, i.e., when $\alpha$95 $\leq$ 10,
weight=1; when $\alpha$95$>$10, weight$<$1. It is also worthwhile to mention
that if samples, where two poles are derived, are exactly from the same place
and same rock, and one A95 is completely inside the other A95, a zero is
assigned as the weight of the data with the larger A95.

For the ``Age error Position to window'' weighting (No $4$ in
Table.~\ref{tab-weit}), if window intersects with young/old end of age bracket
or whole window overlaps with a part of age range, weight= (overlapping part)
/ (age range width); if whole age range is within window, weight= (window width)
/ (age range width) (note that when weight $>1$, it is set back to 1).

The ``Age error Position to window, and $\alpha$95'' weighting (No $5$ in
Table.~\ref{tab-weit}), is a combination of No $3$ and No $4$.


\begin{acknowledgments}
All images are produced using GMT~\cite{W13}.
\end{acknowledgments}

\begin{thebibliography}{}
\bibitem[\protect\citename{Besse \& Courtillot }2002]{B02}
  Besse, J. \& Courtillot, V., 2002, Apparent and true polar wander and the
  geometry of the geomagnetic field over the last 200 Myr, \jgr{}\textbf{107},
  2300.
\bibitem[\protect\citename{Fisher }1953]{F53}
  Fisher, R. A., 1953. Dispersion on a sphere, \textit{Proc. Roy. Soc. London
  Ser. A.}, \textbf{217}, 295\textendash305.
\bibitem[\protect\citename{Jacox \& Samet }2007]{J07}
  Jacox, E. H. \& Samet, H., 2007. Spatial Join Techniques, \textit{ACM Trans.
  Database Syst.}. \textbf{32}, 7.
\bibitem[\protect\citename{King }1955]{K55}
  King, R. F., 1955. The remanent magnetism of artificially deposited sediments,
  \textit{Geophys. Suppl. Mon. Not. Roy. Astron. Soc. Lett.}, \textbf{7},
  115\textendash134.
\bibitem[\protect\citename{M{\"{u}}ller et al. }1993]{M93}
  M{\"{u}}ller, R. D., Royer, J. Y. \& Lawver, L. A., 1993. Revised plate
  motions relative to the hotspots from combined Atlantic and Indian-Ocean
  hotspot tracks, \textit{Geology}, \textbf{21}, 275\textendash278.
\bibitem[\protect\citename{M{\"{u}}ller et al. }1999]{M99}
  M{\"{u}}ller, R. D., Royer, J. Y., Cande, S. C., Roest, W. R. \& Maschenkov,
  S., 1999. New constraints on the Late Cretaceous/Tertiary plate tectonic
  evolution of the Caribbean, \textit{Sedimentary Basins of the World}.
  \textbf{4}, 33\textendash59.
\bibitem[\protect\citename{McQuarrie \& Wernicke }2006]{Mc06}
  McQuarrie, N. \& Wernicke, B. P., 2006. An animated tectonic reconstruction of
  southwestern North America since 36 Ma, \textit{Geosphere}, \textbf{1},
  147\textendash172.
\bibitem[\protect\citename{Pisarevsky }2005]{P05}
  Pisarevsky, S. A., 2005. New edition of the Global Paleomagnetic Database,
  \eos{}\textbf{86}, 170.
\bibitem[\protect\citename{Schettino \& Scotese }2005]{S05}
  Schettino, A. \& Scotese, C. R., 2005. Apparent polar wander paths for the
  major continents (200 Ma to the present day): a palaeomagnetic reference frame
  for global plate tectonic reconstructions, \gji{}\textbf{163},
  727\textendash759.
\bibitem[\protect\citename{Torsvik et al. }1992]{T92}
  Torsvik, T. H., Smethurst, M. A., van der Voo, R., Trench, A., Abrahamsen, N.
  \& Halvorsen, E., 1992. Baltica. A synopsis of Vendian-Permian palaeomagnetic
  data and their palaeotectonic implications, \textit{Earth Sci. Rev.},
  \textbf{33}, 133\textendash152.
\bibitem[\protect\citename{Torsvik \& Smethurst }1999]{T99}
  Torsvik, T. H. \& Smethurst, M. A., 1999. Plate tectonic modelling: virtual
  reality with GMAP, \textit{Comput. Geosci.}, \textbf{25}, 395\textendash402.
\bibitem[\protect\citename{Tauxe \& Kent }2004]{T04}
  Tauxe, L. \& Kent, D. V., 2004. A simplified statistical model for the
  geomagnetic field and the detection of shallow bias in paleomagnetic
  inclinations: Was the ancient magnetic field dipolar? \textit{Geophys.
  Monogr. AGU}, \textbf{145}, 101\textendash115.
\bibitem[\protect\citename{Torsvik et al. }2008]{T08}
  Torsvik, T. H., M{\"{u}}ller, R. D., van der Voo, R., Steinberger, B., \&
  Gaina, C., 2008. Global plate motion frames: Toward a unified model,
  \textit{Rev. Geophys.}, \textbf{46}, RG3004.
\bibitem[\protect\citename{Torsvik et al. }2012]{T12}
  Torsvik, T. H., van der Voo, R., Preeden, U., Mac Niocaill, C., Steinberger,
  B., Doubrovine, P. V., van Hinsbergen, D. J. J., Domeier, M., Gaina, C.,
  Tohver, E., Meert, J. G., McCausland, P. J. A. \& Cocks, L. R. M., 2012.
  Phanerozoic polar wander, palaeogeography and dynamics, \textit{Earth Sci.
  Rev.}, \textbf{114}, 325\textendash368.
\bibitem[\protect\citename{Tauxe et al. }2018]{T18}
  Tauxe L., Banerjee S.K., Butler R.F. \& van der Voo R., 2018.
  \textit{Essentials of Paleomagnetism}, 5th web edn, Available on line
\bibitem[\protect\citename{van der Voo }1990]{v90}
  van der Voo, R., 1990. The reliability of paleomagnetic data,
  \tecto{}\textbf{184}, 1\textendash9.
\bibitem[\protect\citename{Wessel et al. }2013]{W13}
  Wessel, P., Smith, W. H. F., Scharroo, R., Luis, J. \& Wobbe, F.,2013. Generic
  Mapping Tools: Improved version released, \eos{}\textbf{94}, 409\textendash410.
\bibitem[\protect\citename{Young et al. }2018]{Y18}
  Young, A., Flament, N., Maloney, K., Williams, S., Matthews, K., Zahirovic,
  S.
  \& Müller, D., 2018. Global kinematics of tectonic plates and subduction zones
  since the late Paleozoic Era, \textit{Geosci. Front.},
  \textbf{in press}, 000\textendash000.
\end{thebibliography}~\label{lastpage}
