\begin{summary}
Using our APWP similarity measuring tool to find which method\(s\) is$(are)$
good or bad.
\end{summary}

\begin{keywords}
  Moving Average \textendash{} Weighting \textendash{} APWP \textendash{}
  Paleomagnetism.
\end{keywords}

\section{Introduction}

APWPs are generated by combining paleomagnetic poles for a particular rigid
block over the desired age range to produce a smoothed path.

\subsection{Not All Data Are Created Equal}

However, uncertainties in the age and location of paleomagnetic poles can vary
greatly for different poles.

\subsubsection{Age Error}

Although remanent magnetizations are generally assumed to be primary, many
events can cause remagnetisation (in which case the derived pole is `younger'
than the rock). If an event that has occurred since the rock's formation that
should affect the magnetisation (e.g., folding, thermal overprinting due to
intrusion) can be shown to have affected it, then it constrains the
magnetisation to have been acquired before that event. Recognising or ruling
out remagnetisations depends on these field tests, which are not always
performed or possible. Even a passed field test may not be useful if field test
shows magnetisation acquired prior to a folding event tens of millions of years
after initial rock formation.

The most obvious characteristic we can observe from paleomagnetic data is that
some poles have very large age ranges, e.g., more than 100 Myr. The
magnetization age should be some time between the information of the rock and
folding events. There are also others where we have similar position but the
age constraint is much narrower, e.g. 10 Myr window or less. Obviously the
latter kind of data is more valuable than the one with large age range.

\subsubsection{Position Error}

The errors of pole latitudes and longitudes are 95\% confidence ellipses,
which also vary greatly in magnitude. All paleomagnetic poles have some
associated uncertainties due to measurement error and the nature of the
geomagnetic field. More uncertainties can be added by too few samples,
sampling spanning too short a time range to approximate a GAD field, failure
to remove overprints during demagnetisation, etc.

\subsubsection{Data Consistency}

Paleomagnetic poles of a rigid plate or block should be continuous time series.
For a rigid plate, two poles with similar ages shouldn't be dramatically
different in location. Sometimes, this is the case. Sometimes we have further
separated poles with close ages.

There are a number of possible causes for these outliers, including:

\verb"Lithology"

For poor consistency of data, it is potentially because of different
inclinations or declinations. The first thing we should consider about is
their lithology. We want to check if the sample rock are igneous or
sedimentary, because sediment compaction can result in anomalously shallow
inclinations~\cite{T16}. In addition, we also can check if the rock are
redbeds or non-redbeds. Although whether redbeds record a detrital signal or a
later Chemical Remanent Magnetization (CRM) is still somewhat controversial,
both sedimentary rocks and redbeds could lead to inconsistency in direction
compared to igneous rocks.

\verb"Local Rotations"

Local deformation between two paleomagnetic localities invalidates the rigid
plate assumption and could lead to inconsistent VGP directions. So if
discordance is due to local deformation, and we would ideally want to exclude
such poles from our APWP calculation.

\verb"Other Factors"

In most cases, mean pole age (centre of age error) has just been binned. If
any of the poles have large age errors, they could be different ages from each
other and sample entirely different parts of the APWP\@. Conversely, if any of
the poles have too few samples, or were not sampled over enough time to average
to a GAD field, a discordant pole may be due to unreduced secular variation.

\subsubsection{Data Density}

As we go back in time, we have lower quality and lower density (or quantity) of
data, for example, the Precambrian or Early Paleozoic paleomagnetic data are
relatively fewer than Middle-Late Phanerozoic ones, and most of them are not
high-quality, e.g., larger errors in both age and location. The combination of
lower data quality with lower data density means that a single `bad' pole (with
large errors in age and/or location) can much more easily distort the
reconstructed APWP, because there are few or no `good' poles to counteract its
influence.

Data density also varies between different plates. E.g., we have a relatively
high density of paleomagnetic data for North American Craton (NAC), but few
poles exist for Greenland and Arabia. Based on mean age (mean of lower and
upper magnetic ages), for 100\textendash0 Ma, the \textbf{Global Paleomagnetic
Database} (GPMDB) version 4.6b~\cite{P05} has 198 NAC poles, but only 17 for
Greenland and 24 for Arabia.

\subsubsection{Publication Year}

The time when the data was published should also be considered, because
magnetism measuring methodology, technology and equipments have been improved
since the early 20th century. For example, stepwise demagnetisation, which is
the most reliable method of detecting and removing secondary overprints, has
only been in common use since the mid 1980s.

In summary, not all paleomagnetic poles are created equal, which leads to an
important question: how to best combine poles of varying quality into a
coherent and accurate APWP\@?

\subsection{Existing Solutions and General Issues}

Paleomagnetists have proposed a variety of methods to filter so-called ``bad''
data, or give lower weights to those ``bad'' data before generating an APWP,
e.g., two widely used methods: the V90 reliability criteria~\cite{v90} and the
BC02 selection criteria provided by Besse \& Courtillot \shortcite{B02}.
Briefly, the V90 criteria for paleomagnetic results includes seven criteria: (1)
Well determined age; (2) At least 25 samples with Fisher~\cite{F53} precision
$\kappa$ greater than 10 and $\alpha95$ less than 16\degree; (3) Detailed
demagnetisation results reported; (4) Passed field tests; (5) Tectonic coherence
with continent and good structural control; (6) Identified antipodal reversals;
(7) Lack of similarity with younger poles~\cite{T92}. The total criteria
satisfied (0\textendash7) is then used as a measure of a paleomagnetic result's
overall reliability, which is known as Q (quality) factor~\cite{T92}. Q factor
is indeed a very straightforward way to
get a quantitalized reliability score. Also it then can be conveniently used in
the later calculations of APWPs~\cite{T92}. But at the same time this is a
fairly basic filter that lumps together criteria that may not be equally
important. Compared with V90, the BC02 criteria suggests stricter filtering,
e.g., using only poles with at least 6 sampling sites and 36 samples, each site
having $\alpha95$ less than 10\degree\ in the Cenozoic and 15\degree\ in the
Mesozoic. B02 is also straightforward and convenient to use, but some useful
data may be filtered out and wasted especially for a period where there are
only limited number of data. In addition, there has been limited study of how
effective these marking/filtering methods are at reconstructing a `true' APWP,
and for most studies after a basic filtering of `low quality' poles, the
remaining poles are, in fact, treated equally.

Above all, there haven't been any real attempts to study how APWP fits may be
improved by filtering/weighting data. This paper is presented to address these
issues.



\section{Methods}

For most of Earth history, concretely for times before c. 170 Ma, the age of the
oldest magnetic anomaly identification, paleomagnetism is the only accepted
quantitative method for reconstructing plate motions and past paleogeographies.
After about 170 Ma, multiple data sources can help constrain plate motions in
more accurate ways. One of the most developed and studied plate kinematics
models is the Fixed Hotspot Model (FHM)~\cite{M93,M99}, which assumes the
Atlantic and Indian hotspots are relatively fixed. Such a model like the Fixed
Hotspot can predict APWPs for main continents, e.g.\ the North America.

\subsection{Reference Path: The Hotspot Model Predicted}

The oldest pole that can be predicted from the FHM is about 120 Ma. The North
American 120\textendash0 Ma APWP predicted from this rotation model will be
taken as a reference path, which will be compared with paleomagnetic APWPs for
the same plate or continent.

\subsection{120\textendash0 Ma North American Paleomagnetic APWP}

The GPMDB 4.6b~\cite{P05}, data source used here, includes 9514 paleopoles for
ages of 3,500 Ma to the present published from 1925 to 2016. A polygon
can be drawn around a set of data, whose sampling sites we believe belong to a
specific plate or rigid block. Then the {\em Spatial Join\/}
technique~\cite{J07} helps join attributes from the polygon to the paleomagnetic
data based on the spatial relationship allowing data within this polygon to be
extracted from the whole raw large dataset without splitting a subset just for a
specific plate. That allows us to quickly select subsets of the database based
on geographic constraints just as easily as for age. Of course, the boundary of
this polygon must be reasonably along a tectonic boundary (see the details
about data filtering for North America in the Supplementary Material). The
temporal distribution of North American 120\textendash0 Ma poles is shown in
Fig.~\ref{fig-120NAhist}.

\begin{figure*}
\centering
\includegraphics[width=1.01\textwidth]{figures/120NAhist.pdf}
\caption[Distribution of 120\textendash0 Ma North American poles]{Temporal
distribution 120\textendash0 Ma $NAC$ paleomagnetic poles. For distribution a,
each bin only counts the midpoints of pole error bars; For distribution b, as
long as the bar intersects with the bin, it is counted in.
}\label{fig-120NAhist}
\end{figure*}

\subsection{Picking Data for A Certain Time Window}

\subsubsection{Moving Average}

The moving average method, also called ``running mean'' or ``moving
window''~\cite{T08} method, calculates the average of values between a certain
data (age in our case) range; the average is then recalculated as the limits of
the bin are repeatedly incremented upwards. In addition to the traditional
moving windows averaging algorithm, a newly developed moving average method is
also used, referred to here as the ``Age Position Picking'' method. The
difference of this moving average method from the one built in
GMAP~\cite{T99,T08} is that the whole magnetic age range is taken into account
in each window, while GMAP only considers the mid-point of the low and high
magnetic age of each pole, an algorithm referred to as the ``Age Mean Picking''
method.

Normally each VGP in the paleomagnetic database is treated as a point with an
age that is the mid-point between the upper and lower age limits, i.e. ``Age
Mean Picking'', but this is problematic for paleomagnetic data with large age
ranges (especially if they turn out to be primary magnetization that should
plot at old end of age range). We are trying a method (``Age Position Picking'')
that includes a VGP in the moving average bin if any part of its specified age
range falls within that bin. If, for example, we have a pole which is
constrained to within 10 and 20 Ma of age, and we have a 2 Myr moving window
with a 1 Myr age step, then it shouldn't just be in the 14\textendash16 Ma bin
(for the mid-point age of 15 Ma)\textemdash{}it should be in the 9\textendash11,
10\textendash12,11\textendash13,12\textendash14\ldots17\textendash19,
18\textendash20, and 19\textendash21 Ma bins. So the average poles are produced
from each bin, and each original pole is represented over its entire possible
acquisition age. Fig.~\ref{fig-nac-maplat} shows an example of moving average
with a 10 Myr window and a 5 Myr step. So, for example, for the window of 15 Ma
to 5 Ma (the light blue shades in Fig.~\ref{fig-nac-maplat}), the ``Age Mean
Picking'' method calculates the Fisher mean pole of only 8 poles, while the
``Age Position Picking'' method calculates the mean pole of 17 poles. From
comparison of mean poles of the picked poles for the light blue age window with
the two different algorithms (the 10 Ma mean poles of brown paths in
Fig.~\ref{fig-nac-maplat}), the mean pole from ``Age Position Picking'' method
is closer to the 10 Ma pole in the Fixed-Hotpost-Model predicted path.

\begin{figure*}
\centering
\includegraphics[width=1.01\textwidth]{../../../gpmdb/apwp/moving_mean/ma_paleolat_nac_.png}
\caption[Moving average (MA) methods]{An example of 10 Myr moving window and 5
Myr step in the moving average method, based on poles of the $NAC$. Every age
window has a different color. Red points are the midpoints of low and high
magnetic ages. The vertical axis has no specific meaning here.
}\label{fig-nac-maplat}
\end{figure*}

\subsection{Landscape pages}\label{classoptions}

\begin{enumerate}
  \item Use the \verb"table*" or \verb"figure*"
        \verb"\% **"
\begin{verbatim}
% ** \clearpage
% ** \thispagestyle{plate}
% ** \plate{Opposite p.~812, GJI, \textbf{135}}
\begin{figure*}
  \vbox to220mm{\vfil Landscape figure to
                go here. \vfil}
  \caption{}
  \label{landfig}
\end{figure*}
\end{verbatim}
\item the \verb"\pagestyle" command, as follows:
\begin{verbatim}
\documentclass[landscape]{gji}
\pagestyle{empty}
\end{verbatim}
  \item Before each float environment, use the \verb"\setcounter"
\end{enumerate}


\section{Additional facilities}

\begin{enumerate}
  \item Extended commands for specifying a short version of the title and
        author$(s)$ for the running headlines;
  \item A \verb"keywords" environment and a \verb"\nokeywords" command;
 \end{enumerate}

\subsection{Lists}

must be a single line ($\leqslant 45$ characters).
number labelling command after the \verb"\begin{enumerate}". For example, the
list
\begin{enumerate}
\renewcommand{\theenumi}{(\arabic{enumi})}
  \item first item
  \item second item
  \item etc\ldots
\end{enumerate}
was produced by:
\begin{description}
  \item First unnumbered item which has no label and is indented from the left
        margin.
  \item Third unnumbered item.
\end{description}

\subsection{Captions for continued figures and tables}

 \begin{figure}
     \vspace{5.5cm}
     \caption{An example figure in which space has been left for the artwork.}\label{sample-figure}
  \end{figure}

\section[]{Some guidelines for using\\* standard facilities}

\subsection{Sections}

\LaTeX\ provides five levels of section headings and they are all defined in the
\begin{description}
  \item \verb"\subsection"
  \item \verb"\subsubsection"
  \item \verb"\paragraph"
  \item \verb"\subparagraph"
\end{description}

need any other style, see the example in Section~\ref{headings}.

\verb"\\*" to end individual lines and include the optional argument \verb"[]" 

\subsection{Illustrations (or figures)}

\begin{figure*}
 \vspace{5.5cm}
  \caption{An example space has been left for the artwork.}\label{twocol-figure}
\end{figure*}

\verb"figure" environment which would override these decisions. See
`Instructions for Authors' in {\em Geophysical Journal International\/}
the \verb"\caption" command should appear after the figure or space
left for an illustration. For example, Fig.~\ref{sample-figure} is

should be used as in  Fig.~\ref{twocol-figure} using the following commands

\subsection{Tables}

omitted. For example, Table~\ref{symbols} is produced using the
\begin{table}
 \caption{Seismic velocities at major discontinuities.}\label{symbols}
 \begin{tabular}{@{}lcccccc}
  Class & depth & radius
        & $\alpha _{-}$ & $\alpha _{+}$
        & $\beta _{-}$ & $\beta _{+}$ \\
  ICB & 5154 & 1217 & 11.091 & 10.258
        & 3.438 &  0. \\
  CMB & 2889 & 3482 & 8.009 & 13.691
        & 0. & 7.301 \\
 \end{tabular}

 \medskip
subscript $-$ are evaluated just below the discontinuity and those with subscript $+$
\end{table}


\subsection{Running headlines}

three use et~al. The \verb"\pagestyle" and \verb"\thispagestyle"
commands should {\em not\/} be used. Similarly, the commands

\subsection{Typesetting mathematics}

\subsubsection{Displayed mathematics}

provided that you use the \LaTeX\ standard of open and closed square brackets
\[
 \sum_{i=1}^p \lambda_i =
{\mathrm{trace}}(\mathbf{S})
\]
equation, \[ \alpha_{j+1} > \bar{\alpha}+ks_{\alpha} \]
required the command \verb"\eqsecnum" should appear after \verb"begin{document}"

\subsubsection{Bold math italic}\label{boldmathitalic}

The class file provides a font \verb"\mitbf" defined as:
\begin{equation}
  d(\mitbf{s_{t_u}}) = \langle {[RM(\mitbf{x_y} + \mitbf{s_t}) - RM(\mitbf{x_y})]}^2 \rangle
\end{equation}

messages in your log file that read something like ``Warning: Font/shape
`cmm/b/it' in size~\hbox{$< \!\! 9 \!\! >$} not available on input line 649.
Warning: Using external font `cmmi9' instead on input line 649.''


\subsubsection{Bold Greek}\label{boldgreek}

to typeset the equation \[ \mitbf{{\alpha_{\mu}}} = \mitbf{\Theta} \alpha. \]

\subsection{Points to note in formatting text}\label{formtext}

\begin{quote}
\$ \& \% \# \_ \{ and \}
\end{quote}
must be typed
\begin{center}
\begin{quote}
\verb"\$" \verb"\&" \verb"\%" \verb"\#" \verb"\_" \verb"\{" and \verb"\}".
\end{quote}
\end{center}

\texttt{``quoted text.''} they are typed within math commands (\verb"$>$" or \verb"$<$").

\subsubsection{Special symbols}

The macros for the special symbols in Tables~\ref{mathmode} and~\ref{anymode}
%
\begin{table*}
\begin{minipage}{106mm}
\caption{Special symbols which can only be used in math mode.}\label{mathmode}
\begin{tabular}{@{}llllll}
Input & Explanation & Output & Input & Explanation & Output\\
\toprule
\verb"\la"     & less or approx       & $\la$     &
\verb"\ga"     & greater or approx    & $\ga$\\[2pt]
\verb"\getsto" & gets over to         & $\getsto$ &
\verb"\cor"    & corresponds to       & $\cor$\\[2pt]
\verb"\lid"    & less or equal        & $\lid$    &
\verb"\gid"    & greater or equal     & $\gid$\\[2pt]
\verb"\sol"    & similar over less    & $\sol$    &
\verb"\sog"    & similar over greater & $\sog$\\[2pt]
\verb"\lse"    & less over simeq      & $\lse$    &
\verb"\gse"    & greater over simeq   & $\gse$\\[2pt]
\verb"\grole"  & greater over less    & $\grole$  &
\verb"\leogr"  & less over greater    & $\leogr$\\[2pt]
\verb"\loa"    & less over approx     & $\loa$    &
\verb"\goa"    & greater over approx  & $\goa$\\
\bottomrule
\end{tabular}
\end{minipage}
\end{table*}
%
\begin{table*}
\begin{minipage}{115mm}
\caption{Special symbols which don't have to be used in math mode.}\label{anymode}
\begin{tabular}{@{}llllll}
Input & Explanation & Output & Input & Explanation & Output\\
\toprule
\verb"\sun"      & sun symbol            & $\sun$     &
  \verb"\earth"     & earth symbol         & $\earth$   \\[2pt]
\verb"\degr"     & degree                &$\degr$     &
  \verb"\micron"   & \micron{}               & \micron{}    \\[2pt]
  \verb"\diameter" & diameter              & \diameter{}  &
  \verb"\sq"       & square                & \squareforqed\\[2pt]
  \verb"\fd"       & fraction of day       & \fd{}        &
  \verb"\fh"       & fraction of hour      & \fh\\[2pt]
  \verb"\fm"       & fraction of minute    & \fm{}        &
  \verb"\fs"       & fraction of second    & \fs\\[2pt]
  \verb"\fdg"      & fraction of degree    & \fdg{}       &
  \verb"\fp"       & fraction of period    & \fp\\[2pt]
  \verb"\farcs"    & fraction of arcsecond & \farcs{}     &
  \verb"\farcm"    & fraction of arcmin    & \farcm\\[2pt]
  \verb"\arcsec"   & arcsecond             & \arcsec{}    &
  \verb"\arcmin"   & arcminute             & \arcmin\\
\bottomrule
\end{tabular}
\end{minipage}
\end{table*}


\subsection{Bibliography}

approach uses the \verb"\begin{thebibliography}{}" and
\verb"\end{thebibliography}{}" commands.

The second approach uses a simplified scheme using \verb"\begin{references}" and
\verb"\end{references}".

the same author$(s)$ and date, the letters a,b,c, \ldots\

\subsubsection{Biblography References in the text}

\subsubsection{The bibliography}

\begin{enumerate}
  \item if an author has written several papers,
        which, in turn, precede the multi-author papers;
  \item within the two-author paper citations,
\end{enumerate}
%
where \verb"Author(s)" should be commands. \verb"Bibliography entry" should be

\subsubsection{Simplified References and Citations}

brackets \markcite{(Merritt et al., 1996)}, as in \markcite{Ono (1996)}.
\begin{references}
  \reference{}
Azimi, Sh.A., Kalinin, A.Y., Kalinin, V.B., \& Pivovarov, B.L., 1968.
Impulse and transient characteristics of media with linear and  quadratic
absorption laws,
\textit{Izv. Earth Phys.} (English Transl.),
\textbf{2}, 88--93.
\reference{}
Dahlen, F.A., \& Smith, M.L., 1975.
The influence of rotation on the free oscillations of the Earth,
\textit{Phil. Trans. R. Soc. London Ser. A}, \textbf{279}, 143--167.
\reference{}
Durek, J.J., Ritzwoller, M.H., \& Woodhouse, J.H., 1993. Constraining upper
mantle anelasticity using surface wave amplitude anomalies, \gji{} \textbf{114},
249\textendash272.
\end{references}

\subsubsection{Common Journals}

Common journals
\newline
\begin{tabular}{ll}
\verb"\areps" & \areps{} \\
\verb"\epsl"  & \epsl{} \\
\verb"\geo"   & \geo{} \\
\verb"\geop"  & \geop{} \\
\verb"\gji"   & \gji{} \\
\verb"\grl"   & \grl{} \\
\verb"\gsab"  & \gsab{} \\
\verb"\gs"    & \gs{} \\
\verb"\pag"   & \pag{} \\
\verb"\pepi"  & \pepi{} \\
\verb"\rg"    & \rg{} \\
\end{tabular}
%
% The following 2 tables have been moved back in the text to improve page layout
%
\begin{table*}
\begin{minipage}{130mm}
\caption{Authors' notes.}\label{authors}
\begin{tabular}{@{}ll}
\verb"\title[optional short title]{long title}"
                    & short title used in running head\\
\verb"\author[optional short author(s)]{long author(s)}"
                    & short author$(s)$ used in running head\\
\verb"\begin{abstract}...\end{abstract}"& for summary on titlepage\\
\verb"\begin{summary}...\end{summary}"& for abstract on titlepage\\
\verb"\begin{keywords}...\end{keywords}"& for keywords on titlepage\\
\verb"\nokeywords"  & if there are no keywords on titlepage\\
\verb"\begin{figure*}...\end{figure*}" & for a double spanning figure in two-column mode\\
\verb"\begin{table*}...\end{table*}" & for a double spanning table in
                                       two-column mode\\
\verb"\caption*"    & for continuation figure captions\\
\verb"\resetfigno" & resets figures numbers after an appendix\\
\verb"[referee]" & documentclass option for 12/20pt, single col,
                   for manuscript submission\\
\verb"[mreferee]" & documentclass option for 11/17pt, single col,
                   for submission of papers with extensive mathematics\\
\end{tabular}
\end{minipage}
\end{table*}
%
\begin{table*}
\begin{minipage}{130mm}
\caption{Editors' notes.}\label{editors}
\begin{tabular}{@{}lp{270pt}}
\verb"\pagerange{000--000}"& for catchline, note use of en-rule\\
\verb"\pagerange{L00--L00}"& for letters option, used in catchline\\
\verb"\volume{000}" & volume number, for catchline\\
\verb"\pubyear{0000}" & publication year, for catchline\\
\verb"\microfiche{GJI000/0}" & for articles accompanied by microfiche\\
\verb"\journal" & replace the whole catchline at one go\\
\verb"[doublespacing]" & documentclass option for doublespacing\\
\verb"[galley]" & documentclass option for running to galley\\
\verb"[landscape]" & documentclass option for landscape illustrations\\
\verb"[fasttrack]" & documentclass option, for rapid short communications
                   (adds F to folios)\\
\verb"[onecolumn]" & documentclass option for one-column \\
\verb"\bsp"& typesets the final phrase `This paper has been produced
 using the Blackwell Publishing GJI \LaTeX2e\ class file.'\\
\end{tabular}
\end{minipage}
\end{table*}

\subsection{Appendices}

You only need to type \verb"\appendix" once. Thereafter, every \verb"\section"

\section[]{Example of section heading with\\*
  {\mdseries \textsc{S}\lowercase{\textsc{mall}}
  \textsc{C}\lowercase{\textsc{aps}}},
  \lowercase{lowercase},
  \textit{italic}, and bold\\* Greek such as
  $\mitbf{{\mu^\kappa}}$}\label{headings}

This can be built up using text commands and the \verb"mitbf" command introduced

\subsection{Acknowledgments}
with the \texttt{acknowledgments} environment, as

\begin{acknowledgments}
All images are produced using GMT~\cite{W13}.
\end{acknowledgments}

\begin{thebibliography}{}
\bibitem[\protect\citename{Besse \& Courtillot }2002]{B02}
  Besse, J. \& Courtillot, V., 2002, Apparent and true polar wander and the
  geometry of the geomagnetic field over the last 200 Myr, \jgr{}\textbf{107},
  2300.
\bibitem[\protect\citename{Fisher }1953]{F53}
  Fisher, R. A., 1953. Dispersion on a sphere, \textit{Proc. Roy. Soc. London
  Ser. A.}, \textbf{217}, 295\textendash305.
\bibitem[\protect\citename{Jacox \& Samet }2007]{J07}
  Jacox, E. H. \& Samet, H., 2007. Spatial Join Techniques, \textit{ACM Trans.
  Database Syst.}. \textbf{32}, 7.
\bibitem[\protect\citename{M{\"{u}}ller et al. }1993]{M93}
  M{\"{u}}ller, R. D., Royer, J. Y. \& Lawver, L. A., 1993. Revised plate
  motions relative to the hotspots from combined Atlantic and Indian-Ocean
  hotspot tracks, \textit{Geology}, \textbf{21}, 275\textendash278.
\bibitem[\protect\citename{M{\"{u}}ller et al. }1999]{M99}
  M{\"{u}}ller, R. D., Royer, J. Y., Cande, S. C., Roest, W. R. \& Maschenkov,
  S., 1999. New constraints on the Late Cretaceous/Tertiary plate tectonic
  evolution of the Caribbean, \textit{Sedimentary Basins of the World}.
  \textbf{4}, 33\textendash59.
\bibitem[\protect\citename{McQuarrie \& Wernicke }2006]{Mc06}
  McQuarrie, N. \& Wernicke, B. P., 2006. An animated tectonic reconstruction of
  southwestern North America since 36 Ma, \textit{Geosphere}, \textbf{1},
  147\textendash172.
\bibitem[\protect\citename{Pisarevsky }2005]{P05}
  Pisarevsky, S. A., 2005. New edition of the Global Paleomagnetic Database,
  \eos{}\textbf{86}, 170.
\bibitem[\protect\citename{Torsvik et al. }1992]{T92}
  Torsvik, T. H., Smethurst, M. A., van der Voo, R., Trench, A., Abrahamsen, N.
  \& Halvorsen, E., 1992. Baltica. A synopsis of Vendian-Permian palaeomagnetic
  data and their palaeotectonic implications, \textit{Earth Sci. Rev.},
  \textbf{33}, 133\textendash152.
\bibitem[\protect\citename{Torsvik \& Smethurst }1999]{T99}
  Torsvik, T. H. \& Smethurst, M. A., 1999. Plate tectonic modelling: virtual
  reality with GMAP, \textit{Comput. Geosci.}, \textbf{25}, 395\textendash402.
\bibitem[\protect\citename{Torsvik et al. }2008]{T08}
  Torsvik, T. H., M{\"{u}}ller, R. D., van der Voo, R., Steinberger, B., \&
  Gaina, C., 2008. Global plate motion frames: Toward a unified model,
  \textit{Rev. Geophys.}, \textbf{46}, RG3004.
\bibitem[\protect\citename{Tauxe et al. }2016]{T16}
  Tauxe L., Banerjee S.K., Butler R.F. \& van der Voo R., 2016.
  \textit{Essentials of Paleomagnetism}, 4th web edn, Available on line
\bibitem[\protect\citename{van der Voo }1990]{v90}
  van der Voo, R., 1990. The reliability of paleomagnetic data,
  \tecto{}\textbf{184}, 1\textendash9.
\bibitem[\protect\citename{Wessel et al. }2013]{W13}
  Wessel, P., Smith, W. H. F., Scharroo, R., Luis, J. \& Wobbe, F.,2013. Generic
  Mapping Tools: Improved version released, \eos{}\textbf{94}, 409\textendash410.
\bibitem[\protect\citename{Young et al. }2018]{Y18}
  Young, A., Flament, N., Maloney, K., Williams, S., Matthews, K., Zahirovic,
  S.
  \& Müller, D., 2018. Global kinematics of tectonic plates and subduction zones
  since the late Paleozoic Era, \textit{Geosci. Front.},
  \textbf{in press}, 000\textendash000.
\end{thebibliography}


\appendix
\section{For authors}

Table~\ref{authors} is a list of design macros which are unique to GJI\. The

\section{For editors}

The additional features shown in Table~\ref{editors} may be used for production

\bsp{} % ``This paper has been produced using the Blackwell
       %   Publishing GJI \LaTeXe\ class file.''
~\label{lastpage}
