\begin{summary}
Using our APWP similarity measuring tool to find which method is$(are)$ good or
bad.
\end{summary}

\begin{keywords}
  Moving Average \textendash{} Weighting \textendash{} APWP \textendash{}
  Paleomagnetism.
\end{keywords}

\section{Introduction}

APWPs are generated by combining paleomagnetic poles for a particular rigid
block over the desired age range to produce a smoothed path.

\subsection{Not All Data Are Created Equal}

However, uncertainties in the age and location of paleomagnetic poles can vary
greatly for different poles.

\subsubsection{Age Error}

Although remanent magnetizations are generally assumed to be primary, many
events can cause remagnetisation (in which case the derived pole is `younger'
than the rock). If an event that has occurred since the rock's formation that
should affect the magnetisation (e.g., folding, thermal overprinting due to
intrusion) can be shown to have affected it, then it constrains the
magnetisation to have been acquired before that event. Recognising or ruling
out remagnetisations depends on these field tests, which are not always
performed or possible. Even a passed field test may not be useful if field test
shows magnetisation acquired prior to a folding event tens of millions of years
after initial rock formation.

The most obvious characteristic we can observe from paleomagnetic data is that
some poles have very large age ranges, e.g., more than 100 myr. The
magnetization age should be some time between the information of the rock and
folding events. There are also others where we have similar position but the
age constraint is much narrower, e.g. 10 myr window or less. Obviously the
latter kind of data is more valuable than the one with large age range.

\subsubsection{Position Error}

The errors of pole latitudes and longitudes are 95\% confidence ellipses,
which also vary greatly in magnitude. All paleomagnetic poles have some
associated uncertainties due to measurement error and the nature of the
geomagnetic field. More uncertainties can be added by too few samples,
sampling spanning too short a time range to approximate a GAD field, failure
to remove overprints during demagnetisation, etc.

\subsubsection{Data Consistency}

Paleomagnetic poles of a rigid plate or block should be continuous time series.
For a rigid plate, two poles with similar ages shouldn't be dramatically
different in location. Sometimes, this is the case. Sometimes we have further
separated poles with close ages.

There are a number of possible causes for these outliers, including:

\verb"Lithology"

For poor consistency of data, it is potentially because of different
inclinations or declinations. The first thing we should consider about is
their lithology. We want to check if the sample rock are igneous or
sedimentary, because sediment compaction can result in anomalously shallow
inclinations~\cite{T16}. In addition, we also can check if the rock are
redbeds or non-redbeds. Although whether redbeds record a detrital signal or a
later Chemical Remanent Magnetization (CRM) is still somewhat controversial,
both sedimentary rocks and redbeds could lead to inconsistency in direction
compared to igneous rocks.

\verb"Local Rotations"

Local deformation between two paleomagnetic localities invalidates the rigid
plate assumption and could lead to inconsistent VGP directions. So if
discordance is due to local deformation, and we would ideally want to exclude
such poles from our APWP calculation.

\verb"Other Factors"

In most cases, mean pole age (centre of age error) has just been binned. If
any of the poles have large age errors, they could be different ages from each
other and sample entirely different parts of the APWP\@. Conversely, if any of
the poles have too few samples, or were not sampled over enough time to average
to a GAD field, a discordant pole may be due to unreduced secular variation.

\subsubsection{Data Density}

As we go back in time, we have lower quality and lower density (or quantity) of
data, for example, the Precambrian or Early Paleozoic paleomagnetic data are
relatively fewer than Middle-Late Phanerozoic ones, and most of them are not
high-quality, e.g., larger errors in both age and location. The combination of
lower data quality with lower data density means that a single `bad' pole (with
large errors in age and/or location) can much more easily distort the
reconstructed APWP, because there are few or no `good' poles to counteract its
influence.

Data density also varies between different plates. E.g., we have a relatively
high density of paleomagnetic data for North American Craton (NAC), but few
poles exist for Greenland and Arabia. Based on mean age (mean of lower and
upper magnetic ages), for 100\textendash0 Ma, GPMDB 4.6b~\cite{P05} has 198 NAC
poles, but only 17 for Greenland and 24 for Arabia.

\subsubsection{Publication Year}

The time when the data was published should also be considered, because
magnetism measuring methodology, technology and equipments have been improved
since the early 20th century. For example, stepwise demagnetisation, which is
the most reliable method of detecting and removing secondary overprints, has
only been in common use since the mid 1980s.

In summary, not all paleomagnetic poles are created equal, which leads to an
important question: how to best combine poles of varying quality into a
coherent and accurate APWP\@?

\subsection{Existing Solutions and General Issues}

Paleomagnetists have proposed a variety of methods
to filter so-called ``bad'' data, or give lower weights to those ``bad'' data
before generating an APWP, e.g., two widely used methods: the V90 reliability
criteria~\cite{v90} and the BC02 selection criteria provided by Besse \&
Courtillot \shortcite{B02}. Briefly, the V90 criteria for paleomagnetic results
includes seven criteria: (1) Well determined age; (2) At least 25 samples with
Fisher~\cite{F53} precision $\kappa$ greater than 10 and $\alpha95$ less than
16\degree; (3) Detailed demagnetisation results reported; (4) Passed field
tests; (5) Tectonic coherence with continent and good structural control; (6)
Identified antipodal reversals; (7) Lack of similarity with younger
poles~\cite{T92}. The total criteria satisfied (0\textendash7) is then used as
a measure of a paleomagnetic result's overall reliability, which is known as Q
(quality) factor~\cite{T92}. Q factor is indeed a very straightforward way to
get a quantitalized reliability score. Also it then can be conveniently used in
the later calculations of APWPs~\cite{T92}. But at the same time this is a
fairly basic filter that lumps together criteria that may not be equally
important. Compared with V90, the BC02 criteria suggests stricter filtering,
e.g., using only poles with at least 6 sampling sites and 36 samples, each site
having $\alpha95$ less than 10\degree\ in the Cenozoic and 15\degree\ in the
Mesozoic. B02 is also straightforward and convenient to use, but some useful
data may be filtered out and wasted especially for a period where there are
only limited number of data. In addition, there has been limited study of how
effective these marking/filtering methods are at reconstructing a `true' APWP,
and for most studies after a basic filtering of `low quality' poles, the
remaining poles are, in fact, treated equally.

Above all, there haven't been any real attempts to study how APWP fits may be
improved by filtering/weighting data. This paper is presented to address these
issues.



\section{Methods}

The \textbf{Global Paleomagnetic Database} (GPMDB) Version 4.6b~\cite{P05},
data source used here, includes 9360 paleopoles for ages of 3,500 Ma to the
present published from 1925 to January 2011. A polygon can be drawn around a set
of data, whose sampling sites we believe belong to a specific plate or rigid
block. Then the {\em Spatial Join\/} technique~\cite{J07} helps join attributes
from the polygon to the paleomagnetic data based on the spatial relationship
allowing data within this polygon to be extracted from the whole raw large
dataset without splitting a subset just for a specific plate. That allows us to
quickly select subsets of the database based on geographic constraints just as
easily as for age. Of course, the boundary of this polygon must be reasonably
along a tectonic boundary.

\subsection{120\textendash0 Ma North America APWP}

Plate ID 101 polygon in the recently published Plate Model~\cite{M16},
including its children 108 and 109 polygons for 120\textendash0 Ma, is used to select the
sampling sites of the paleopoles for North America. According to the plate model
rotation data~\cite{M16}, the Avalon/Acadia block (108) is fixed to 101 during the
geologic period from Cretaceous to the present day. The Piedmont block (109) is
also fixed to 101 since about 300 Ma~\cite{C14}. Then in order to be compared
with the Fix Hotspot Model (120\textendash0 Ma)~\cite{M93,M99}, the paleopoles with age
ranging 120\textendash0 Ma are further selected through constraining the lower magnetic
age ``LOMAGAGE <=120''. According to a published plate kinematic model of
southwestern North America since 36 Ma~\cite{Mc06}, the paleopoles constrained
by some western terranes whose Plate IDs are also 101 and those that are fixed to
101 during 120\textendash0 Ma shown in the plate model rotation data~\cite{M16} except
some constrained by a small terrane 101 are removed. So finally there are four
polygons total for this Plate Model~\cite{M16}. According to the plate
kinematic model of southwestern North America~\cite{Mc06}, the poles with age
younger than 10 Ma located within the western 101 terrane (RANGE_ID=74 in the
plate kinematic model~\cite{Mc06}) right south of the smallest 101 one should
be included.


\subsection{Additional document class options}\label{classoptions}


\subsection{Landscape pages}

\begin{enumerate}
  \item Use the \verb"table*" or \verb"figure*" environments in your
        need to key in the three lines which are marked with \verb"\% **",
\begin{verbatim}
% ** \clearpage
% ** \thispagestyle{plate}
% ** \plate{Opposite p.~812, GJI, \textbf{135}}
\begin{figure*}
  \vbox to220mm{\vfil Landscape figure to
                go here. \vfil}
  \caption{}
  \label{landfig}
\end{figure*}
\end{verbatim}
\item the \verb"landscape" the \verb"\pagestyle" command, as follows:
\begin{verbatim}
\documentclass[landscape]{gji}
\pagestyle{empty}
\end{verbatim}
  \item using the \verb"table*" and \verb"figure*"
  \item Before each float environment, use the \verb"\setcounter"
\begin{verbatim}
\setcounter{table}{0}
\begin{table*}
 \begin{minipage}{115mm}
 \caption{Images of global seismic tomography.}
 \label{tab1}
 \begin{tabular}{@{}llllcll}
   :
 \end{tabular}
 \end{minipage}
\end{table*}
\end{verbatim}
The corresponding example for a figure would be:
\begin{verbatim}
\clearpage
\setcounter{figure}{12}
\begin{figure*}
 \vspace{144mm}
 \caption{Travel times for regional model.}
 \label{fig13}
\end{figure*}
\end{verbatim}
\end{enumerate}


\section{Additional facilities}

\begin{enumerate}
  \item Extended commands for specifying a short version of the title and
        author$(s)$ for the running headlines;
  \item A \verb"keywords" environment and a \verb"\nokeywords" command;
 \end{enumerate}

\subsection{Titles and author's name}

must be a single line ($\leqslant 45$ characters). Moreover, the main
\begin{verbatim}
\title[Geophys.\ J.\ Int.:
       \LaTeXe\ Guide for Authors]
  {Geophysical Journal International:
   \LaTeXe\ style guide for authors}
\end{verbatim}
and
\begin{verbatim}
\author[B.L.N. Kennett]
   {B.L.N. Kennett$^1$
  \thanks{Pacific Region Office, GJI} \\
  $^{1}$Research School of Earth Sciences,
    Australian National University,
    Canberra ACT \emph{0200}, Australia
  }
\end{verbatim}
The \verb"\thanks" note produces a footnote to the title or author.

\subsection{Key words and Summary}

usual way using the \verb"\maketitle" command. Immediately following
words. The summary should be enclosed within an \verb"summary"
\verb"keywords" environment. For example, the titles for this guide
\begin{verbatim}
\maketitle
\begin{summary}
 This guide is for authors who are preparing
 papers for \textit{Geophysical Journal
 International} using the \LaTeXe\ document
 preparation system and the GJI style file.
\end{summary}
\begin{keywords}
 \LaTeXe\ -- class files: \verb"gji.cls"\ --
 sample text -- user guide.
\end{keywords}

\section{Introduction}
  :
\end{verbatim}
The heading `\textbf{Key words}' is included automatically and the key words are
should insert the \verb"\nokeywords" command immediately after the end of the
\verb"summary" or \verb"abstract" environment. This ensures that the vertical
space after the abstract and/or title is correct and that any \verb"thanks"
\begin{verbatim}
\maketitle
\begin{abstract}
  :
\end{abstract}
\nokeywords

\section{Introduction}
  :
\end{verbatim}

Note that the \verb"summary" and \verb"abstract" environments have the same
effect for the documentclass \verb"gji.cls"

\subsection{Lists}

The GJI style provides numbered lists using the \verb"enumerate" environment and
unnumbered lists using the \verb"description" environment with an empty label.
Bulleted lists are not part of the GJI style and the \verb"itemize" environment
should not be used.

The enumerated list numbers each list item with roman numerals:
\begin{enumerate}
  \item first item
  \item second item
  \item third item
\end{enumerate}
number labelling command after the \verb"\begin{enumerate}". For example, the
list
\begin{enumerate}
\renewcommand{\theenumi}{(\arabic{enumi})}
  \item first item
  \item second item
  \item etc\ldots
\end{enumerate}
was produced by:
\begin{verbatim}
\begin{enumerate}
 \renewcommand{\theenumi}{(\arabic{enumi})}
  \item first item
       :
\end{enumerate}
\end{verbatim}
Unnumbered lists are provided using the \verb"description" environment. For
example,
\begin{description}
  \item First unnumbered item which has no label and is indented from the left
        margin.
  \item Second unnumbered item.
  \item Third unnumbered item.
\end{description}
was produced by,
\begin{verbatim}
\begin{description}
 \item First unnumbered item...
 \item Second unnumbered item.
 \item Third unnumbered item.
\end{description}
\end{verbatim}

\subsection{Captions for continued figures and tables}

two floats; the second float should use the \verb"caption*" command with a
suitable caption:
\begin{verbatim}
\begin{table}
 \caption*{-- \textit{continued}}
  \begin{tabular}{@{}lccll}
  :
  \end{tabular}
\end{table}
\end{verbatim}

 \begin{figure}
     \vspace{5.5cm}
     \caption{An example figure in which space has been left for the artwork.}\label{sample-figure}
  \end{figure}

\section[]{Some guidelines for using\\* standard facilities}

\subsection{Sections}

\LaTeX\ provides five levels of section headings and they are all defined in the
GJI style file:
\begin{description}
  \item \verb"\section"
  \item \verb"\subsection"
  \item \verb"\subsubsection"
  \item \verb"\paragraph"
  \item \verb"\subparagraph"
\end{description}

Section numbers are given for section, subsection, subsubsection and paragraph
need any other style, see the example in Section~\ref{headings}.

If you find your section/subsection (etc.)\ headings are wrapping round, you
must use the \verb"\\*" to end individual lines and include the optional
argument \verb"[]" in the section command. This ensures that the turnover is
flushleft.

\subsection{Illustrations (or figures)}

\begin{figure*}
 \vspace{5.5cm}
  \caption{An example figure spanning two-columns in which space has been left
  for the artwork.}\label{twocol-figure}
\end{figure*}

\verb"figure" environment which would override these decisions. See
`Instructions for Authors' in {\em Geophysical Journal International\/}
the \verb"\caption" command should appear after the figure or space
left for an illustration. For example, Fig.~\ref{sample-figure} is
produced using the following commands:
\begin{verbatim}
\begin{figure}
 \vspace{5.5cm}
 \caption{An example figure in which space has
          been left for the artwork.}
 \label{sample-figure}
\end{figure}
\end{verbatim}

should be used as in  Fig.~\ref{twocol-figure} using the following commands
\begin{verbatim}
\begin{figure*}
 \vspace{5.5cm}
   \caption{An example figure spanning two-columns
     in which space has been left for the artwork.}
   \label{twocol-figure}
\end{figure*}
\end{verbatim}

\subsection{Tables}

\verb"table" environment which would override these decisions. Table
captions should be at the top, therefore the \verb"\caption" command

The \verb"tabular" environment can be used to produce tables with
following ways:
\begin{enumerate}
  \item additional vertical space is inserted on either side of a rule;
  \item vertical lines are not produced.
\end{enumerate}
Commands to redefine quantities such as \verb"\arraystretch" should be
omitted. For example, Table~\ref{symbols} is produced using the
following commands.
\begin{table}
 \caption{Seismic velocities at major discontinuities.}\label{symbols}
 \begin{tabular}{@{}lcccccc}
  Class & depth & radius
        & $\alpha _{-}$ & $\alpha _{+}$
        & $\beta _{-}$ & $\beta _{+}$ \\
  ICB & 5154 & 1217 & 11.091 & 10.258
        & 3.438 &  0. \\
  CMB & 2889 & 3482 & 8.009 & 13.691
        & 0. & 7.301 \\
 \end{tabular}

 \medskip
 The ICB represents the boundary between the inner and outer cores and
subscript $-$ are evaluated just below the discontinuity and
those with subscript $+$ are evaluated just above the discontinuity.
\end{table}
\begin{verbatim}
\begin{table}
 \caption{Seismic velocities at major
          discontinuities.}
 \label{symbols}
 \begin{tabular}{@{}lcccccc}
  Class & depth & radius
        & $\alpha _{-}$ & $\alpha _{+}$
        & $\beta _{-}$ & $\beta _{+}$ \\
  ICB & 5154 & 1217 & 11.091 & 10.258
        & 3.438 &  0. \\
  CMB & 2889 & 3482 & 8.009 & 13.691
        & 0. & 7.301 \\
 \end{tabular}

 \medskip
 The ICB represents the boundary ...
... evaluated just above the discontinuity.

\end{table}
\end{verbatim}

If you have a table that is to extend over two columns, you need to use
\verb"table*" in a minipage environment, i.e., you can say
\begin{verbatim}
\begin{table*}
\begin{minipage}{80mm}
 \caption{Caption which will wrap round to the
          width of the minipage environment.}
 \begin{tabular}{%
      :
 \end{tabular}
\end{minipage}
\end{table*}
\end{verbatim}

\subsection{Running headlines}

three use et~al. The \verb"\pagestyle" and \verb"\thispagestyle"
commands should {\em not\/} be used. Similarly, the commands
\verb"\markright" and \verb"\markboth" should not be necessary.

\subsection{Typesetting mathematics}

\subsubsection{Displayed mathematics}

provided that you use the \LaTeX\ standard of open and closed square brackets
\[
 \sum_{i=1}^p \lambda_i =
{\mathrm{trace}}(\mathbf{S})
\]
was typeset in the GJI style using the commands
\begin{verbatim}
\[
 \sum_{i=1}^p \lambda_i =
{\mathrm{trace}}(\mathbf{S})
\]
\end{verbatim}
equation, \[ \alpha_{j+1} > \bar{\alpha}+ks_{\alpha} \]
which was (wrongly) typeset using double dollars as follows:
\begin{verbatim}
$$ \alpha_{j+1} > \bar{\alpha}+ks_{\alpha} $$
\end{verbatim}
Note that \verb"\mathrm" will produce a roman character in math mode.

For numbered equations use the \verb"equation" and \verb"eqnarray" environments
which will give the correct positioning. If equation numbering by section is
required the command \verb"\eqsecnum" should appear after \verb"begin{document}"

\subsubsection{Bold math italic}\label{boldmathitalic}

The class file provides a font \verb"\mitbf" defined as:
\begin{verbatim}
\newcommand{\mitbf}[1]{
  \hbox{\mathversion{bold}$#1$}}
\end{verbatim}
Which can be used as follows, to typset the equation
\begin{equation}
  d(\mitbf{s_{t_u}}) = \langle {[RM(\mitbf{x_y} + \mitbf{s_t}) - RM(\mitbf{x_y})]}^2 \rangle
\end{equation}
the input should be
\begin{verbatim}
\begin{equation}
  d(\mitbf{s_{t_u}}) = \langle {[RM(\mitbf{x_y}
  + \mitbf{s_t}) - RM(\mitbf{x_y})]}^2 \rangle
\end{equation}
\end{verbatim}

messages in your log file that read something like ``Warning: Font/shape
`cmm/b/it' in size~\hbox{$< \!\! 9 \!\! >$} not available on input line 649.
Warning: Using external font `cmmi9' instead on input line 649.'' If you have


\subsubsection{Bold Greek}\label{boldgreek}

To get bold Greek you use the same method as for bold math italic. Thus you can
input
\begin{verbatim}
\[ \mitbf{{\alpha_{\mu}}} =
\mitbf{\Theta} \alpha. \]
\end{verbatim}
to typeset the equation \[ \mitbf{{\alpha_{\mu}}} = \mitbf{\Theta} \alpha. \]


\subsection{Points to note in formatting text}\label{formtext}

\begin{quote}
\$ \& \% \# \_ \{ and \}
\end{quote}
must be typed
\begin{center}
\begin{quote}
\verb"\$" \verb"\&" \verb"\%" \verb"\#" \verb"\_" \verb"\{" and \verb"\}".
\end{quote}
\end{center}

\LaTeX\ interprets all double quotes as closing quotes. Therefore quotation
\texttt{``quoted text.''}

they are typed within math commands (\verb"$>$" or \verb"$<$").

\subsubsection{Special symbols}

The macros for the special symbols in Tables~\ref{mathmode} and~\ref{anymode}
%
\begin{table*}
\begin{minipage}{106mm}
\caption{Special symbols which can only be used in math mode.}\label{mathmode}
\begin{tabular}{@{}llllll}
Input & Explanation & Output & Input & Explanation & Output\\
\toprule
\verb"\la"     & less or approx       & $\la$     &
\verb"\ga"     & greater or approx    & $\ga$\\[2pt]
\verb"\getsto" & gets over to         & $\getsto$ &
\verb"\cor"    & corresponds to       & $\cor$\\[2pt]
\verb"\lid"    & less or equal        & $\lid$    &
\verb"\gid"    & greater or equal     & $\gid$\\[2pt]
\verb"\sol"    & similar over less    & $\sol$    &
\verb"\sog"    & similar over greater & $\sog$\\[2pt]
\verb"\lse"    & less over simeq      & $\lse$    &
\verb"\gse"    & greater over simeq   & $\gse$\\[2pt]
\verb"\grole"  & greater over less    & $\grole$  &
\verb"\leogr"  & less over greater    & $\leogr$\\[2pt]
\verb"\loa"    & less over approx     & $\loa$    &
\verb"\goa"    & greater over approx  & $\goa$\\
\bottomrule
\end{tabular}
\end{minipage}
\end{table*}
%
\begin{table*}
\begin{minipage}{115mm}
\caption{Special symbols which don't have to be used in math mode.}\label{anymode}
\begin{tabular}{@{}llllll}
Input & Explanation & Output & Input & Explanation & Output\\
\toprule
\verb"\sun"      & sun symbol            & $\sun$     &
  \verb"\earth"     & earth symbol         & $\earth$   \\[2pt]
\verb"\degr"     & degree                &$\degr$     &
  \verb"\micron"   & \micron{}               & \micron{}    \\[2pt]
  \verb"\diameter" & diameter              & \diameter{}  &
  \verb"\sq"       & square                & \squareforqed\\[2pt]
  \verb"\fd"       & fraction of day       & \fd{}        &
  \verb"\fh"       & fraction of hour      & \fh\\[2pt]
  \verb"\fm"       & fraction of minute    & \fm{}        &
  \verb"\fs"       & fraction of second    & \fs\\[2pt]
  \verb"\fdg"      & fraction of degree    & \fdg{}       &
  \verb"\fp"       & fraction of period    & \fp\\[2pt]
  \verb"\farcs"    & fraction of arcsecond & \farcs{}     &
  \verb"\farcm"    & fraction of arcmin    & \farcm\\[2pt]
  \verb"\arcsec"   & arcsecond             & \arcsec{}    &
  \verb"\arcmin"   & arcminute             & \arcmin\\
\bottomrule
\end{tabular}
\end{minipage}
\end{table*}

The command \verb"\chemical" is provided to set chemical species with an even
level for subscripts (not produced in standard mathematics mode). Thus
\verb"\chemical{Fe_{2}^{2+}Cr_{2}O_{4}}" will produce
\chemical{Fe_{2}^{2+}Cr_{2}O_{4}}.


\subsection{Bibliography}

Two methods are provided for managing citations and references. The first
approach uses the \verb"\begin{thebibliography}{}" and
\verb"\end{thebibliography}{}" commands.

The second approach uses a simplified scheme using \verb"\begin{references}" and
\verb"\end{references}" commands.

e.g. Draine (1978) or (Begelman, Blandford \& Rees 1984). Where more than one
reference is cited having the same author$(s)$ and date, the letters a,b,c,
\ldots\ should follow the date; e.g.\ Smith (1988a), Smith (1988b), etc. The

\subsubsection{Biblography References in the text}

\textendash{} \verb"\cite{key}" \textendash{} to produce the author and date,
and another form \textendash{} \verb"\shortcite{key}" \textendash{} which
produces the date only. Thus, Rutherford \& Hawker \shortcite{rh} is produced by
\begin{verbatim}
Rutherford \& Hawker \shortcite{rh}
\end{verbatim}
while~\cite{hi} is produced by
\begin{verbatim}
\cite{hi}
\end{verbatim}

\subsubsection{The bibliography}

\begin{enumerate}
  \item if an author has written several papers, some with other authors, the
        which, in turn, precede the multi-author papers;
  \item within the two-author paper citations, the order is determined by the
  \item within the multi-author paper citations, the order is chronological,
\end{enumerate}
%
Each entry takes the form
\begin{verbatim}
\bibitem[\protect\citename{Author(s), }%
  Date]{tag} Bibliography entry
\end{verbatim}
where \verb"Author(s)" should be the author names as they are cited in the text,
\verb"Date" is the date to be cited in the text, and \verb"tag" is the tag that
is to be used as an argument for the \verb"\cite{}" and \verb"\shortcite{}"
commands. \verb"Bibliography entry" should be the material that is to appear in
the bibliography, suitably formatted.

\subsubsection{Simplified References and Citations}

brackets \markcite{(Merritt et al., 1996)} or you may mention the author as part
of your sentence and include only the year in brackets, as in \markcite{Ono
(1996)}.

\begin{verbatim}
\begin{references}
\reference
Azimi, Sh.A., Kalinin, A.Y., Kalinin, V.B., \&
  Pivovarov, B.L., 1968.
Impulse and transient characteristics of media with
  linear and quadratic
absorption laws,
\textit{Izv. Earth Phys.} (English Transl.),
\textbf{2}, 88--93.
\reference
Dahlen, F.A., \& Smith, M.L., 1975.
The influence of rotation on the free oscillations
  of the Earth,
\textit{Phil. Trans. R. Soc. London Ser. A},
\textbf{279}, 143--167.
\reference
Durek, J.J., Ritzwoller, M.H., \& Woodhouse, J.H.,
  1993.
Constraining upper mantle anelasticity using
  surface wave amplitude anomalies,
\gji, \textbf{114}, 249--272.
\end{references}
\end{verbatim}
produces the reference list
\begin{references}
  \reference{}
Azimi, Sh.A., Kalinin, A.Y., Kalinin, V.B., \& Pivovarov, B.L., 1968.
Impulse and transient characteristics of media with linear and  quadratic
absorption laws,
\textit{Izv. Earth Phys.} (English Transl.),
\textbf{2}, 88--93.
\reference{}
Dahlen, F.A., \& Smith, M.L., 1975.
The influence of rotation on the free oscillations of the Earth,
\textit{Phil. Trans. R. Soc. London Ser. A}, \textbf{279}, 143--167.
\reference{}
Durek, J.J., Ritzwoller, M.H., \& Woodhouse, J.H., 1993. Constraining upper
mantle anelasticity using surface wave amplitude anomalies, \gji{} \textbf{114},
249\textendash272.
\end{references}

\subsubsection{Common Journals}

Common journals
\newline
\begin{tabular}{ll}
\verb"\areps" & \areps{} \\
\verb"\bssa"  & \bssa{} \\
\verb"\eos"   & \eos{}  \\
\verb"\eps"   & \eps{} \\
\verb"\epsl"  & \epsl{} \\
\verb"\gca"   & \gca{} \\
\verb"\geo"   & \geo{} \\
\verb"\geop"  & \geop{} \\
\verb"\gji"   & \gji{} \\
\verb"\gjras" & \gjras{} \\
\verb"\grl"   & \grl{} \\
\verb"\gsab"  & \gsab{} \\
\verb"\gs"    & \gs{} \\
\verb"\jgr"   & \jgr{} \\
\verb"\jseis" & \jseis{} \\
\verb"\mnras" & \mnras{} \\
\verb"\pag"   & \pag{} \\
\verb"\pepi"  & \pepi{} \\
\verb"\rg"    & \rg{} \\
\verb"\tecto" & \tecto{} \\
\end{tabular}
%
% The following 2 tables have been moved back in the text to improve page layout
%
\begin{table*}
\begin{minipage}{130mm}
\caption{Authors' notes.}\label{authors}
\begin{tabular}{@{}ll}
\verb"\title[optional short title]{long title}"
                    & short title used in running head\\
\verb"\author[optional short author(s)]{long author(s)}"
                    & short author$(s)$ used in running head\\
\verb"\begin{abstract}...\end{abstract}"& for summary on titlepage\\
\verb"\begin{summary}...\end{summary}"& for abstract on titlepage\\
\verb"\begin{keywords}...\end{keywords}"& for keywords on titlepage\\
\verb"\nokeywords"  & if there are no keywords on titlepage\\
\verb"\begin{figure*}...\end{figure*}" & for a double spanning figure in two-column mode\\
\verb"\begin{table*}...\end{table*}" & for a double spanning table in
                                       two-column mode\\
\verb"\caption*"    & for continuation figure captions\\
\verb"\resetfigno" & resets figures numbers after an appendix\\
\verb"[referee]" & documentclass option for 12/20pt, single col,
                   for manuscript submission\\
\verb"[mreferee]" & documentclass option for 11/17pt, single col,
                   for submission of papers with extensive mathematics\\
\end{tabular}
\end{minipage}
\end{table*}
%
\begin{table*}
\begin{minipage}{130mm}
\caption{Editors' notes.}\label{editors}
\begin{tabular}{@{}lp{270pt}}
\verb"\pagerange{000--000}"& for catchline, note use of en-rule\\
\verb"\pagerange{L00--L00}"& for letters option, used in catchline\\
\verb"\volume{000}" & volume number, for catchline\\
\verb"\pubyear{0000}" & publication year, for catchline\\
\verb"\microfiche{GJI000/0}" & for articles accompanied by microfiche\\
\verb"\journal" & replace the whole catchline at one go\\
\verb"[doublespacing]" & documentclass option for doublespacing\\
\verb"[galley]" & documentclass option for running to galley\\
\verb"[landscape]" & documentclass option for landscape illustrations\\
\verb"[fasttrack]" & documentclass option, for rapid short communications
                   (adds F to folios)\\
\verb"[onecolumn]" & documentclass option for one-column \\
\verb"\bsp"& typesets the final phrase `This paper has been produced
 using the Blackwell Publishing GJI \LaTeX2e\ class file.'\\
\end{tabular}
\end{minipage}
\end{table*}

\subsection{Appendices}

\begin{verbatim}
\appendix
\section{For authors}
     :
\section{For editors}
\end{verbatim}
You only need to type \verb"\appendix" once. Thereafter, every \verb"\section"
command will generate a new appendix which will be numbered A, B, etc.

\section[]{Example of section heading with\\*
  {\mdseries \textsc{S}\lowercase{\textsc{mall}}
  \textsc{C}\lowercase{\textsc{aps}}},
  \lowercase{lowercase},
  \textit{italic}, and bold\\* Greek such as
  $\mitbf{{\mu^\kappa}}$}\label{headings}

This can be built up using text commands and the \verb"mitbf" command introduced
above

\begin{verbatim}
\section[]{Example of section heading with\\*
  {\mdseries \textsc{S}\lowercase{\textsc{mall}}
  \textsc{C}\lowercase{\textsc{aps}}},
  \lowercase{lowercase},
  \textit{ italic}, and bold\\* Greek such as
  $\mitbf{{\mu^\kappa}}$}\label{headings}
\end{verbatim}

\subsection{Acknowledgments}
Acknowledgments after the main text and before the appendices can be included
with the \texttt{acknowledgments} environment, as
\begin{verbatim}
\begin{acknowledgments}
We wish to thank ...
\end{acknowledgments}
\end{verbatim}

\begin{acknowledgments}
All images are produced using GMT~\cite{W13}.
\end{acknowledgments}

\begin{thebibliography}{}
  \bibitem[\protect\citename{Butcher }1992]{bu}
    Butcher J. 1992. \textit{Copy-editing: The Cambridge Handbook}, 3rd edn,
    Cambridge Univ. Press, Cambridge.
  \bibitem[\protect\citename{The Chicago Manual }%
    1982]{cm} \textit{The Chicago Manual of Style}, Univ. Chicago Press,
    Chicago, 1982.
  \bibitem[\protect\citename{Fisher }1953]{F53}
    Fisher, R. A., 1953. Dispersion on a sphere,
    \textit{Proc. Roy. Soc. London Ser. A.}, \textbf{217}, 295\textendash305.
  \bibitem[\protect\citename{Torsvik et al. }1992]{T92}
	Torsvik, T. H., Smethurst, M. A., van der Voo, R., Trench, A., Abrahamsen,
	N. \& Halvorsen, E., 1992. Baltica. A synopsis of Vendian-Permian
	palaeomagnetic data and their palaeotectonic implications, \textit{Earth
	Sci. Rev.}, \textbf{33}, 133\textendash152.
  \bibitem[\protect\citename{Wessel et al. }2013]{W13}
	Wessel, P., Smith, W. H. F., Scharroo, R., Luis, J. \& Wobbe, F.,2013.
	Generic Mapping Tools: Improved version released, \textit{Eos. Trans. AGU},
	\textbf{94}, 409\textendash410.
  \bibitem[\protect\citename{Hinderer }1986]{hi}
    Hinderer, J., 1986. Resonance effects of the earth's fluid core in earth
    rotation, in \textit{Solved and Unsolved Problems}, pp. 277\textendash296,
    ed. Cazenave A., Reidel, Dordrecht.
  \bibitem[\protect\citename{Kopka \& Daly}1995]{kd}
    Kopka H. \& Daly P.W., 1995, \textit{A guide to} \LaTeX2e,
    Addison\textendash{}Wesley, New York
  \bibitem[\protect\citename{Tauxe et al. }2016]{T16}
	Tauxe L., Banerjee S.K., Butler R.F. \& van der Voo R., 2016.
	\textit{Essentials of Paleomagnetism}, 4th web edn, Available on line
  \bibitem[\protect\citename{Lindberg }1986]{bl}
    Lindberg, C., 1986. Multiple taper harmonic analysis of terrestrial free
    oscillations, \textit{PhD thesis}, University of California.
  \bibitem[\protect\citename{van der Voo }1990]{v90}
    van der Voo, R., 1990. The reliability of paleomagnetic data,
	\textit{Tectonophysics}. \textbf{184}, 1\textendash9.
  \bibitem[\protect\citename{Pisarevsky }2005]{P05}
	Pisarevsky, S. A., 2005. New edition of the Global Paleomagnetic Database,
	\textit{Eos. Trans. AGU}. \textbf{86}, 170.
  \bibitem[\protect\citename{Jacox \& Samet }2007]{J07}
	Jacox, E. H. \& Samet, H., 2007. Spatial Join Techniques, \textit{ACM
	Trans. Database Syst.}. \textbf{32}, 7.
  \bibitem[\protect\citename{Besse \& Courtillot }%
    2002]{B02}  Besse, J. \& Courtillot, V., 2002, Apparent and true polar
    wander and the geometry of the geomagnetic field over the last 200 Myr,
   \textit{J. Geophys. Res.}, \textbf{107}, 2300
\end{thebibliography}


\appendix
\section{For authors}

Table~\ref{authors} is a list of design macros which are unique to GJI\. The
list displays each macro's name and description.

\section{For editors}

The additional features shown in Table~\ref{editors} may be used for production
purposes.

\bsp{} % ``This paper has been produced using the Blackwell
       %   Publishing GJI \LaTeXe\ class file.''
~\label{lastpage}

