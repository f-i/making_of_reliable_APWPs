\section{Supplementary Materials}

\subsection{How to Constrain Plate 801's Data}
Plate ID 801 polygon in the recently published Plate Model~\cite{Y18}, including
its children 675 (Sumba block) and 684 (Timor block) polygons for
120\textendash0 Ma (Fig.~\ref{fig_AUpolygon}), is used to select the sampling
sites of the paleopoles for Australia (Fig.~\ref{fig_AUpmdata}). According to
the plate model rotation data~\cite{Y18}, 675 and 684 are fixed to 101 during
the geologic period from c.145 Ma to the present.

\begin{figure}
\includegraphics[scale=.32]{figures/801children.pdf}
\caption{The Plate ID 801 polygons (yellow) with its children 675 and 684
(yellow with labels) used to constrain the 120\textendash0 Ma data for
Australia.}\label{fig_AUpolygon}
\end{figure}

On the southeast of the main Australia plate (the largest polygon in
Fig.~\ref{fig_AUpolygon}), there is a triangle-shaped small polygon 850
(Tasmania block) which is fixed to 801 since c.100 Ma according to
the~\cite{Y18} rotation data. With that attribute, 805 contributes more data
younger than c.100Ma for the later analysis. Ultimately the final 99 extracted
datasets is shown in Fig.~\ref{fig_AUfinal}.

\begin{figure}
\includegraphics[scale=.32]{figures/801children_2.pdf}
\caption{All the 120\textendash0 Ma data constrained by the Plate ID 801, 675
and 684 polygons for Australia.}\label{fig_AUpmdata}
\end{figure}

\begin{figure}
\includegraphics[scale=.32]{figures/801children_3.pdf}
\caption{The final filtered datasets (red stars) for later analysis on
Australia. The Plate ID 850 helps increase the amount of qualified datasets for
100\textendash0 Ma.}\label{fig_AUfinal}
\end{figure}

\subsection{How to Constrain Plate 501's Data}
(Fig.~\ref{fig_INpolygon}), is used to select the sampling
sites of the paleopoles for India (Fig.~\ref{fig_INpmdata}).

\begin{figure}
\includegraphics[scale=.35]{figures/501polygon120_0.pdf}
\caption{The Plate ID 501 polygons (yellow) used to constrain the
120\textendash0 Ma data for India.}\label{fig_INpolygon}
\end{figure}

\begin{figure}
\includegraphics[scale=.35]{figures/501sites.pdf}
\caption{All the 120\textendash0 Ma data constrained by the Plate ID 501
polygons for India.}\label{fig_INpmdata}
\end{figure}

Also based on this model of the tectonic interactions between India, Arabia and
Asia since the Jurassic~\cite{G15} (Fig.~\ref{fig_INrfd}), part of the
paleopoles constrained by the north two small terranes whose Plate IDs are also
501 in fact had gone through regional rotations and here are removed. So finally
75 datasets are left (Fig.~\ref{fig_INfinal}). Spatially Indian paleomagnetic
data are more evenly distributed on the India plate than North American and
Australian above.

\begin{figure}
\includegraphics[scale=.35]{figures/501rift_fault_detachment.pdf}
\caption{The rifts, faults and detachments (blue lines) around India, used to
filter out those data that might be influenced by local tectonic
rotations.}\label{fig_INrfd}
\end{figure}

\begin{figure}
\includegraphics[scale=.34]{figures/501final.pdf}
\caption{The final filtered datasets (red stars) for 120\textendash0 Ma India.
Those poles that had been influenced by local tectonic rotations are shown as
white dots.}\label{fig_INfinal}
\end{figure}~\label{lastpage}
