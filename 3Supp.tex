\section{Supplementary Materials}

\subsection{How to Constrain Plate 101's Data}
The data-constraining polygons are from the recently published plate
model~\cite{Y18} (Fig.~\ref{fig_NApolygon}).

\begin{figure}
\includegraphics[scale=.32]{figures/101108109polygon120_0.pdf}
\caption{The Plate ID 101 polygons (yellow) with its children 108 and 109
(yellow with labels) used to constrain the 120\textendash0 Ma data for North
America.}\label{fig_NApolygon}
\end{figure}

The 101, 108 and 109 polygons constrained 191 datasets (Fig.~\ref{fig_NApmdata}).

\begin{figure}
\includegraphics[scale=.32]{figures/101108109sites.pdf}
\caption{All the 120\textendash0 Ma data constrained by the Plate ID 101, 108
and 109 polygons for North America.}\label{fig_NApmdata}
\end{figure}

A published rotation dataset~\cite{Mc06} (Fig.~\ref{fig_NAwUS}) is used to
justify some poles from west US areas back to actual positions.

\begin{figure}
\includegraphics[scale=.32]{figures/101108109wUS.pdf}
\caption{The west US polygons (blue) used to constrain the data that might be
  influenced by local tectonic rotations.}\label{fig_NAwUS}
\end{figure}

\subsection{2}
2

\subsection{3}
3

\begin{itemize}
\item 11
\item 22
\item 33
\item 44
\end{itemize}

\paragraph{Special cases} 55

\subsection{4}
4

\begin{thebibliography}{}
\bibitem[\protect\citename{McQuarrie \& Wernicke }2006]{Mc06}
  McQuarrie, N. \& Wernicke, B. P., 2006. An animated tectonic reconstruction of
  southwestern North America since 36 Ma, \textit{Geosphere}, \textbf{1},
  147\textendash172.
\bibitem[\protect\citename{Young et al. }2018]{Y18}
  Young, A., Flament, N., Maloney, K., Williams, S., Matthews, K., Zahirovic,
  S.
  \& Müller, D., 2018. Global kinematics of tectonic plates and subduction zones
  since the late Paleozoic Era, \textit{Geosci. Front.},
  \textbf{in press}, 000\textendash000.
\end{thebibliography}\label{lastpage}
