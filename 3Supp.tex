\section{Supplementary Materials}

\subsection{How to Constrain Plate 101's Data}
The data-constraining polygons are from the recently published plate
model~\cite{Y18} (Fig.~\ref{fig_NApolygon}). Plate ID 101 polygon in the
recently published Plate Model~\cite{Y18}, including its children 108
(Avalon/Acadia block) and 109 (Piedmont block) polygons for 120\textendash0 Ma,
is used to select the sampling sites of the paleopoles for North America.
According to the plate model rotation data~\cite{Y18}, 108 is fixed to 101
during the geologic period from Cretaceous to the present day. 109 is also fixed
to 101 since about 300 Ma~\cite{C14}. Then in order to be compared with the FHM
(120\textendash0 Ma)~\cite{M93,M99}, the paleopoles with age ranging
120\textendash0 Ma are further selected through constraining the lower magnetic
age ``LOMAGAGE $<=$ 120''. In addition, the RESULTNO=6007 dataset should also be
included according to a published plate kinematic model~\cite{Mc06}
(Fig.~\ref{fig_NAwUS}), although it is in the PlateID=178 polygon. In the end,
191 datasets in total are extracted (Fig.~\ref{fig_NApmdata}).

\begin{figure}
\includegraphics[scale=.32]{figures/101108109polygon120_0.pdf}
\caption{The Plate ID 101 polygons (yellow) with its children 108 and 109
(yellow with labels) used to constrain the 120\textendash0 Ma data for North
America.}\label{fig_NApolygon}
\end{figure}

Also based on this model of southwestern North America since 36 Ma~\cite{Mc06}
(Fig.~\ref{fig_NAwUS}), part of the paleopoles constrained by the four small
western terranes whose Plate IDs are also 101 in fact had gone through regional
rotations and here are removed. However, the poles with age younger than 10 Ma
located within the largest western 101 terrane (on the south of the smallest
western 101 terrane; corresponding to the RANGE\_ID=74 polygon in the
model~\cite{Mc06}) should be included. So finally 133 of the 191 datasets remain
(Fig.~\ref{fig_NAfinal}). Spatially North American paleomagnetic data are mainly
from the western and eastern margins of the plate.

\begin{figure}
\includegraphics[scale=.32]{figures/101108109sites.pdf}
\caption{All the 120\textendash0 Ma data constrained by the Plate ID 101, 108
and 109 polygons for North America.}\label{fig_NApmdata}
\end{figure}

\begin{figure}
\includegraphics[scale=.32]{figures/101108109wUS.pdf}
\caption{The west US polygons (blue) used to constrain the data that might be
  influenced by local tectonic rotations.}\label{fig_NAwUS}
\end{figure}

\begin{figure}
\includegraphics[scale=.35]{figures/101108109final.pdf}
\caption{The final filtered datasets (red stars) for later analysis. Those
poles that had been influenced by local tectonic rotations are shown as white
dots.}\label{fig_NAfinal}
\end{figure}

\subsection{2}
2

\subsection{3}
3

\begin{itemize}
\item 11
\item 22
\item 33
\item 44
\end{itemize}

\paragraph{Special cases} 55

\subsection{4}
4

\begin{thebibliography}{}
\bibitem[\protect\citename{Christeson et al. }2014]{C14}
  Christeson, G. L., Van Avendonk, H. J. A., Norton, I. O., Snedden, J. W.,
  Eddy, D. R., Karner, G. D. \& Johnson, C. A., 2014. Deep crustal structure in
  the eastern Gulf of Mexico, \textit{J. Geophys. Res. Solid Earth},
  \textbf{401}, 183\textendash195.
\bibitem[\protect\citename{M{\"{u}}ller et al. }1993]{M93}
  M{\"{u}}ller, R. D., Royer, J. Y. \& Lawver, L. A., 1993. Revised plate
  motions relative to the hotspots from combined Atlantic and Indian-Ocean
  hotspot tracks, \textit{Geology}, \textbf{21}, 275\textendash278.
\bibitem[\protect\citename{M{\"{u}}ller et al. }1999]{M99}
  M{\"{u}}ller, R. D., Royer, J. Y., Cande, S. C., Roest, W. R. \& Maschenkov,
  S., 1999. New constraints on the Late Cretaceous/Tertiary plate tectonic
  evolution of the Caribbean, \textit{Sedimentary Basins of the World}.
  \textbf{4}, 33\textendash59.
\bibitem[\protect\citename{McQuarrie \& Wernicke }2006]{Mc06}
  McQuarrie, N. \& Wernicke, B. P., 2006. An animated tectonic reconstruction of
  southwestern North America since 36 Ma, \textit{Geosphere}, \textbf{1},
  147\textendash172.
\bibitem[\protect\citename{Young et al. }2018]{Y18}
  Young, A., Flament, N., Maloney, K., Williams, S., Matthews, K., Zahirovic,
  S.
  \& Müller, D., 2018. Global kinematics of tectonic plates and subduction zones
  since the late Paleozoic Era, \textit{Geosci. Front.},
  \textbf{in press}, 000\textendash000.
\end{thebibliography}\label{lastpage}
